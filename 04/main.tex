\documentclass[12pt]{amsart}
\usepackage{amsmath}
\usepackage{amsthm}
\usepackage{amsfonts}
\usepackage{amssymb}
\usepackage[margin=1in]{geometry}
\usepackage{hyperref}
\hypersetup{
    colorlinks=true,
    linkcolor=blue
}

\theoremstyle{definition}
\newtheorem{theorem}{Theorem}[section]
\newtheorem{lemma}[theorem]{Lemma}
\newtheorem{definition}[theorem]{Definition}
\newtheorem{corollary}[theorem]{Corollary}
\newtheorem{proposition}[theorem]{Proposition}
\newtheorem{conjecture}[theorem]{Conjecture}
\newtheorem{remark}[theorem]{Remark}
\newtheorem{example}[theorem]{Example}
\newtheorem{problem}[theorem]{Problem}
\newtheorem{notation}[theorem]{Notation}
\newtheorem{question}[theorem]{Question}
\newtheorem{caution}[theorem]{Caution}

\begin{document}

\title{Quiz 4}

\maketitle

\begin{enumerate}
	\item Can you solve $2^x = 5 \mod 7$? If so, do it. If not, why?
	\item Is the following true or false: $\ln x = \mathcal O(\sqrt{x})$.
	\item Let's say the message space, ciphertext space, and the key space is $\mathbb{F}_{11}$ 
		and you know that messages are being encrypted by 
		\begin{displaymath}
			e_k(m) = km \mod 11
		\end{displaymath}
		Suppose you discover that for $m=2$ that $e_k(m) = 4$. What is $k$? 
\end{enumerate}
\end{document}
